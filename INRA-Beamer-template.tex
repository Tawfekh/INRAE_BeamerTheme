\documentclass[compress,10pt]{beamer}
% version imprimable pour assistance
%\documentclass[10pt, green, handout]{beamer}
\usepackage[T1]{fontenc}
\usepackage[default]{raleway}
\usepackage[utf8]{inputenc}
\usepackage[frenchb]{babel} % le document est en français
\usepackage{amsmath}
\usepackage{graphicx}       % pour insérer des figures
\usepackage{xcolor}         % pour définir plus de couleurs
\usetheme{INRA}  %Applique le theme INRA (ce dernier doit être present dans le repertoire courant)
%-------------------------------------------------------------------------------
% Quelques options pdf
%-------------------------------------------------------------------------------
\hypersetup{
pdfpagemode = FullScreen, % afficher le pdf en plein écran
pdfauthor   = {},%
pdftitle    = {},%
pdfsubject  = {},%
pdfkeywords = {Science,Impact},%
pdfcreator  = {PDFLaTeX,emacs,AucTeX},%
pdfproducer = {INRA}%
}

%-------------------------------------------------------------------------------
\title{INRA Beamer Theme}
\subtitle{Sous-titre facultatif}
\author{Nom de l'auteur}
%-------------------------------------------------------------------------------
\begin{document}

%-F1------------------------------------------------------------------------------
\begin{frame}
  \titlepage
\end{frame}
\section{Debut}
%-F2------------------------------------------------------------------------------
\begin{frame}{Première page}
%----------------------------------------------------------------------
\begin{itemize}
\item Ce thème reste simple
\item Il ne définit que le outertheme et le colortheme
\item Vous pouvez essayer d'ajouter d'autres options selon vos
  habitudes et préférences
\end{itemize} 
{\onslide<2->
\begin{alertblock}{Attention}
  Ce thème n'a été que peu testé, certaines combinaisons de styles ou
  thèmes pourraient ne pas permettre la compilation ...ou simplement
  rendre votre présentation hideuse.
\end{alertblock}
}
\end{frame}

\begin{frame}[fragile]{Modifications}
\begin{block}{Conseils}
  \begin{enumerate}
  \item Testez de façon incrémentalle votre code
  \item Faites appel au thème INRA en dernier.
  \end{enumerate}
\end{block}

{\onslide<2->
\begin{exampleblock}{Ainsi}
\begin{verbatim}
\useinnertheme[shadow=true]{rounded}
\usetheme{INRA} 
\end{verbatim}
\end{exampleblock}
}
\end{frame}

\section{Fin}
%-F3------------------------------------------------------------------------------
\begin{frame}{Seconde page}
%----------------------------------------------------------------------
\begin{itemize}
\item <1-> $\LaTeX$ permet surtout un très bon rendu des équations
\item <2-> $ y= \mu + \sum^{n}_{i=1} \alpha_i + e  $
\end{itemize}  
\end{frame}

\end{document}

%%%Local Variables:
%%% mode: latex
%%% eval: (TeX-PDF-mode 1)
%%% ispell-local-dictionary: "francais"
%%% eval: (flyspell-mode 1)
%%%End:


